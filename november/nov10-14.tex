\input{../header.tex}

\title{Work Log for November 10th-14th}
\author{Logan Brown}
%\date{} % Activate to display a given date or no date (if empty),
         % otherwise the current date is printed 

\begin{document}
\maketitle
%\tableofcontents


\section{Goals for the Week}
%Paste output from writeWeek
\begin{enumerate}
	\item RC example
	\item Load external R packages, use them with RC
\end{enumerate}

\section{Progress/Notes}

\subsection{RC example}

The RC example calculates the line of best fit for the collatz length of 1 to 1000. Here's how the output will typically look

\begin{verbatim}
[lbrown@star1 RC]$ make exec
../../EXTLIB/MPICH/mpich-shared/bin/mpirun -n 1 ./driver RCcar.cfg
Initializing the executive...
rank 0 going to statically run 'RC'
Executing --file=script.r
Line of Best Fit: y=0.0322921x + 44.3798
Standard Deviation = 40.87291
Graph written to CollatzGraph.pdf 
Data written to CollatzData.csv 
\end{verbatim}

And here's CollatzGraph.pdf.

\includepdf[pages={1}]{data/CollatzGraph.pdf}

\subsection{Load external R packages, use them with RC}


\subsection{Documentation}

\subsubsection{Document the RC car}
Just comment the code

Added in a section at the beginning describing the code. 
\begin{verbatim}
/********************************************************************
 * Written by Logan Brown
 * NICS and UTK, 2014
 *
 * This code defines an IEL module function that executes 'script.r'
 * using the R language interpreter. Script.r must be placed in the
 * directory where the driver code is executed.
 *
 * iel-2.0/CASES/RC has an example using the Collatz Conjecture
 *
********************************************************************/
\end{verbatim}

And added a similar section to the R script
\begin{verbatim}
############################################################################
# Script written by Logan Brown
# NICS and UT Knoxville, 2014
# This script finds the length the sequences generated by the following rule
#
#    Let a_0 be an integer. For all n > 0
#       a_(n+1) = (a_n)/2        if a_n is even; 
#       a_(n+1) = 3(a_n) + 1     if a_n is odd and a_n != 1
#    When a_n = 1 the sequence stops (otherwise it cycles 1, 4, 2, 1, 4...)
#
# The script finds the length of these sequences for the first 1000 inputs
# by default, with the number of inputs determined by "number". It then
# finds a line of best fit for the increasing sequence length, and gets
# the standard deviation of the sequence lengths.
# It also writes out the calculated data to a file, and generates a graph.
############################################################################
\end{verbatim}




\subsubsection{Ising Model}
Write up a paragraph on the Ising Model -- Done. See Ising.pdf

\includepdf[pages={1}]{ising.pdf}

\subsection{Read through paper for Kwai}

I made some notes. Do we want to talk about Kraken? 

\section{Goals for next Week}
\begin{enumerate}
\item PSUADE
\item DAKOTA
\item Further documentation?
\end{enumerate}


\end{document} %End of day document, REMOVE
