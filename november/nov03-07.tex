\input{../header.tex}

\title{Work Log for November 3rd-7th}
\author{Logan Brown}
%\date{} % Activate to display a given date or no date (if empty),
         % otherwise the current date is printed 

\begin{document}
\maketitle
%\tableofcontents


\section{Goals for the Week}
\begin{enumerate}
\item PSUADE for the DIEL
\item R library in DIEL
\item C calling R in the DIEL
\end{enumerate}

\section{Progress/Notes}

\subsection{PSUADE for the DIEL}

\subsection{R for the DIEL}

\subsubsection{R libraries}

\begin{verbatim}
./configure --enable-R-shlib --prefix=/home/lbrown/iel-2.0/EXTLIB/R/inst
make
make install
\end{verbatim}



\subsubsection{C calling R}

I added a folder to the iel called RC.

The key points of rc.c are...

\begin{verbatim}
#include <R.h>
#include <Rinternals.h>
#include <Rembedded.h>
#include <stdio.h>
#include <string.h>

extern int R_running_as_main_program;   /* in ../unix/system.c */
.
.
.
	R_running_as_main_program = 0;
	Rf_initialize_R(3, arguments);
	Rf_mainloop();
\end{verbatim}

The rest is merely a matter of parsing and generating the command line arguments for Rf\_initialize\_R(3, arguments)

I'm working on getting an example together now.


\subsubsection{R standalone Math library}

\begin{verbatim}
/home/lbrown/iel-2.0/EXTLIB/R/R-3.1.2/src/nmath/standalone
\end{verbatim}

All of the R math functions are actually written in C. This directory has all of those c functions. For better performance, we may want to link the R standalone math library.


\section{Goals for next Week}
\begin{enumerate}
\item RC example
\item Load external R packages, use them with RC
\end{enumerate}


\end{document} %End of day document, REMOVE
