% !TEX TS-program = pdflatex
% !TEX encoding = UTF-8 Unicode

% This is a simple template for a LaTeX document using the "article" class.
% See "book", "report", "letter" for other types of document.

\documentclass[11pt]{article} % use larger type; default would be 10pt


\usepackage{ulem}
\newcommand\NoIndent[1]{%
  \par\vbox{\parbox[t]{\linewidth}{#1}}%
}


\usepackage[utf8]{inputenc} % set input encoding (not needed with XeLaTeX)

%%% Examples of Article customizations
% These packages are optional, depending whether you want the features they provide.
% See the LaTeX Companion or other references for full information.

%%% PAGE DIMENSIONS
\usepackage{geometry} % to change the page dimensions
\geometry{a4paper} % or letterpaper (US) or a5paper or....
% \geometry{margin=2in} % for example, change the margins to 2 inches all round
% \geometry{landscape} % set up the page for landscape
%   read geometry.pdf for detailed page layout information

\usepackage{graphicx} % support the \includegraphics command and options

% \usepackage[parfill]{parskip} % Activate to begin paragraphs with an empty line rather than an indent

%%% PACKAGES
\usepackage{booktabs} % for much better looking tables
\usepackage{array} % for better arrays (eg matrices) in maths
\usepackage{paralist} % very flexible & customisable lists (eg. enumerate/itemize, etc.)
\usepackage{verbatim} % adds environment for commenting out blocks of text & for better verbatim
\usepackage{subfig} % make it possible to include more than one captioned figure/table in a single float
% These packages are all incorporated in the memoir class to one degree or another...

%%% HEADERS & FOOTERS
\usepackage{fancyhdr} % This should be set AFTER setting up the page geometry
\pagestyle{fancy} % options: empty , plain , fancy
\renewcommand{\headrulewidth}{0pt} % customise the layout...
\lhead{}\chead{}\rhead{}
\lfoot{}\cfoot{\thepage}\rfoot{}

%%% SECTION TITLE APPEARANCE
\usepackage{sectsty}
\allsectionsfont{\sffamily\mdseries\upshape} % (See the fntguide.pdf for font help)
% (This matches ConTeXt defaults)

%%% ToC (table of contents) APPEARANCE
\usepackage[nottoc,notlof,notlot]{tocbibind} % Put the bibliography in the ToC
\usepackage[titles,subfigure]{tocloft} % Alter the style of the Table of Contents
\renewcommand{\cftsecfont}{\rmfamily\mdseries\upshape}
\renewcommand{\cftsecpagefont}{\rmfamily\mdseries\upshape} % No bold!

%%% END Article customizations


\usepackage{verbatim}
\usepackage{amsmath}


\begin{document}

\section{Parallelization}

The NSE version of the code is taking too long, and is not using much parallelization. I want to look into more ways that we can use parallelization to improve the performance of the code. 

\subsection{Chop up the Genome}
According to Cedric and Mike, Wei Chen has done this in the past, and only sees significant improvement up to 6 or 8 processes. I don't see the parallel code anywhere in the current code (removed?) 

\subsection{Chop up the C code}
The C code is taking up $\approx40\%$ of the runtime, largely because it has to exponentiate and normalize a huge amount of data. Mike doesn't want to start initializing new threads at the C level, which is understandable. Of course we could start the threads earlier, but we don't want them to just be idling until the C code.

\subsection{Use Parallelization on the MCMC}
We could take multiple steps with the MCMC using different processors for each step. Multiple proposal values will let us converge faster, but we'll get diminishing returns on adding more processors. I can't imagine getting much advantage beyond 3 processors or so.

\section{The DIEL}

The only way I see the cubfits package going into the DIEL is with some kind of parallel tempering being applied. The module won't really want to exchange with any other modules, and doesn't need any sensitivity analysis or optimization (at this point).

As a note: any application of Mike's code to the DIEL will definitely require R to be a compatible part of the DIEL.

\subsection{Parallel Tempering}
The cubfits code is an MCMC. So we could use the same sort of parallel tempering that the Ising model uses. The performance could be improved by asynchronus parallel exchanges. This would be especially good when we move to the cubappr model (with no $\phi$ values), and we have to propose parameters from the beginning. We'll definitely be running multiple chains.



\end{document}
