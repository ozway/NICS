% !TEX TS-program = pdflatex
% !TEX encoding = UTF-8 Unicode

% This is a simple template for a LaTeX document using the "article" class.
% See "book", "report", "letter" for other types of document.

\documentclass[11pt]{article} % use larger type; default would be 10pt


\usepackage{ulem}
\newcommand\NoIndent[1]{%
  \par\vbox{\parbox[t]{\linewidth}{#1}}%
}


\usepackage[utf8]{inputenc} % set input encoding (not needed with XeLaTeX)

%%% Examples of Article customizations
% These packages are optional, depending whether you want the features they provide.
% See the LaTeX Companion or other references for full information.

%%% PAGE DIMENSIONS
\usepackage{geometry} % to change the page dimensions
\geometry{a4paper} % or letterpaper (US) or a5paper or....
% \geometry{margin=2in} % for example, change the margins to 2 inches all round
% \geometry{landscape} % set up the page for landscape
%   read geometry.pdf for detailed page layout information

\usepackage{graphicx} % support the \includegraphics command and options

% \usepackage[parfill]{parskip} % Activate to begin paragraphs with an empty line rather than an indent

%%% PACKAGES
\usepackage{booktabs} % for much better looking tables
\usepackage{array} % for better arrays (eg matrices) in maths
\usepackage{paralist} % very flexible & customisable lists (eg. enumerate/itemize, etc.)
\usepackage{verbatim} % adds environment for commenting out blocks of text & for better verbatim
\usepackage{subfig} % make it possible to include more than one captioned figure/table in a single float
% These packages are all incorporated in the memoir class to one degree or another...

%%% HEADERS & FOOTERS
\usepackage{fancyhdr} % This should be set AFTER setting up the page geometry
\pagestyle{fancy} % options: empty , plain , fancy
\renewcommand{\headrulewidth}{0pt} % customise the layout...
\lhead{}\chead{}\rhead{}
\lfoot{}\cfoot{\thepage}\rfoot{}

%%% SECTION TITLE APPEARANCE
\usepackage{sectsty}
\allsectionsfont{\sffamily\mdseries\upshape} % (See the fntguide.pdf for font help)
% (This matches ConTeXt defaults)

%%% ToC (table of contents) APPEARANCE
\usepackage[nottoc,notlof,notlot]{tocbibind} % Put the bibliography in the ToC
\usepackage[titles,subfigure]{tocloft} % Alter the style of the Table of Contents
\renewcommand{\cftsecfont}{\rmfamily\mdseries\upshape}
\renewcommand{\cftsecpagefont}{\rmfamily\mdseries\upshape} % No bold!

%%% END Article customizations


\usepackage{verbatim}
\usepackage{amsmath}






\title{Work Log for August}
\author{Logan Brown}
%\date{} % Activate to display a given date or no date (if empty),
         % otherwise the current date is printed 

\begin{document}
\maketitle
%\tableofcontents


\setcounter{section}{02} %week number minus 1
\setcounter{subsection}{-1}
\setcounter{subsubsection}{0}

\section{Week of August 18th-22nd}
\subsection{Goals for the Week}
\textit{Goals for the Week}
\begin{enumerate}
\item Ubuntu Installation
\item Install PSUADE
\item Install DAKOTA

\end{enumerate}

\subsection{Progress / Notes}
\subsubsection{Ubuntu}
Advantages: Faster than the VM or cygwin, better for installing packages. More likely to succeed when trying things such as the DAKOTA Library Mode, or when adding the PSUADE onto the DIEL.

Disadvantages: Took some time to finish the installation. will be a brief adjustment.

The install was successful. All of my work logs and important information was cloud saved, in dropbox and skydrive, so it should all be readily accessible.


\subsubsection{PSUADE}

Installed psuade using the instructions in psuade.tex / psuade.pdf

\begin{enumerate}
\item tar xzvf PSUADE.tar.gz
\item cd PSUADE\_1.6.0
\item mkdir build; cd build
\item cmake .. \&$>$ cmake.output \&
\item make \&$>$ make.output \&
\end{enumerate}

Also set up the PATH and LD\_LIBRARY\_PATH

\subsubsection{DAKOTA}
I have all the prerequisites except for boost\_regex.so.1.49.0

sudo apt-get install libboost-all-dev got boost\_regex.so.1.53.0

\subsection{Goals for next Week}
\begin{enumerate}
\item Put these worklogs in the GitHub and send to Kwai
\item Get DAKOTA running
\item See if I can get Source Dakota running (do not devote toooo much time to this)
\item Get some hard data from a PSUADE example
\end{enumerate}


\end{document} %End of document
